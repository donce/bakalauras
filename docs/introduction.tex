Klasifikavimas - tai dažnai sutinkama užduotis, turintį įvairių sprendimo būdų.
Šios uždavinio tikslas - identifikuoti, kuriai grupei priklauso tiriamas objektas.
Tiriamieji objektai dažniausiai būna vienos rūšies, aprašomi tam tikrais parametrais, o grupės, kuriems jie yra priskiriami - iš anksto žinomos.
Pavyzdžiui, galima klasifikuoti gyvūnus pagal tam tikras jų fizines savybes - kojų ilgį, storį, kitas kūno apimtis, kailio ilgį ir pan.
Natūralu, kad kiekvienas net ir tos pačios rūšies gyvūnas turės šiek tiek kitokius parametrus, tačiau šie parametrai dažniausiai turi įvairius dėsningumus, pagal kuriuos galima bandyti atspėti, kuriai rūšiai tam tikras gyvūnas priklauso.


Norint išspręsti konkretų klasifikavimo uždavinį, akivaizdžiausias sprendimas galėtų būti šių grupių parametrų ištyrimas - pavyzdžiui, norint mokėti atskirti triušius nuo liūtų turint jų ūgius nėra sunki užduotis.
Tačiau problema kyla, kai atskiriamos klasės yra labai panašios viena į kitą - tokiu atveju pastebėti tam tikrus dėsningumus ir juos sumodeliuoti bei realizuoti ir kur kas sunkiau.
Be to, sprendžiant konkretų klasifikavimo uždavinį, tektų gilintis į klasifikuojamus objektus - pavyzdžiui, norint sukurti tam tikrų kiškių rūšių klasifikavimą, gilios žinios apie šias kiškių rūšių savybes būtų privalomos.
Dažniausiai įvairūs dėsningumai apima ne vieną dydį, bet jų kombinaciją, kurią atrasti ir apskaičiuoti reikalautų daug pastangų.
O norint efektyviai atskirti tam tikras objektų klases, gali prireikti daugybės skirtingų dėsningumų.
Šios priežastys labai apsunkina efektyvaus klasifikavimo algoritmo kūrimą analizuojant klases, todėl yra retai naudojamas.

Vienas populiariausių klasifikavimo sprendimo metodų - klasifikavimas naudojant neuroninį tinklą.
Turint pakankamai didelį tiriamų objektų duomenų kiekį, galima įžvelgti tam tikrus dėsningumus.
Neuroninis tinklas leidžia tai automatizuoti - vietoje to, kad žmogus mokytųsi apie objekto savybes, tai atliekama su neuroniniu tinklu.
Turimi duomenys panaudojami apmokyti neuroninį tinklą, kuris po to geba pats pasakyti, kuriai grupei tiriamas objektas priklauso.
Žinoma, neuroninis tinklas nėra visada teisus, kadangi jis remiasi patirtimi, kurią įgijo pavyzdinių duomenų apmokymo metu, tačiau jei šie apmokymui buvo surinkta pakankamai korektiškų duomenų, neuroninis tinklas geba klasifikuoti objektus pakankamai tiksliai.

Dimensiškumo mažinimas (angl.~\textit{dimensionality reduction})
Savybių ištraukimas (angl.~\textit{feature extraction})

\TODO{TODO: pridėti paaiškinimą, kad dažniausiai turime daug duomenų?}

\TODO{neuroniniai tinklai?}
\TODO{dimensiškumo mažinimas}
\TODO{tyrimo tikslai?}
