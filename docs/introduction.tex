Klasifikavimas - tai dažnai sutinkama užduotis, turintį įvairių sprendimo būdų.
Šios uždavinio tikslas - identifikuoti, kuriai grupei priklauso tiriamas objektas.
Tiriamieji objektai dažniausiai būna vienos rūšies, aprašomi tam tikrais parametrais, o grupės, kuriems jie yra priskiriami - iš anksto žinomos.
Pavyzdžiui, galima klasifikuoti gyvūnus pagal tam tikras jų fizines savybes - kojų ilgį, storį, kitas kūno apimtis, kailio ilgį ir pan.
Natūralu, kad kiekvienas net ir tos pačios rūšies gyvūnas turės šiek tiek kitokius parametrus, tačiau šie parametrai dažniausiai turi įvairius proporcingumus, pagal kuriuos galima bandyti atspėti, kuriai rūšiai tam tikras gyvūnas priklauso.

%TODO: pridėti paaiškinimą, kad dažniausiai turime daug duomenų?

Norint išspręsti konkretų klasifikavimo uždavinį, paprasčiausias sprendimas atrodo galėtų būti šių grupių parametrų ištyrimas - pavyzdžiui, norint mokėti atskirti triušius nuo liūtų turint jų ilgius nėra sunki užduotis.
Tačiau problema kyla, kai atskiriamos klasės yra labai panašios viena į kitą - tokiu atveju pastebėti tam tikrus dėsningumus ir juos sumodeliuoti bei realizuoti ir kur kas sunkiau.
Be to, sprendžiant konkretų klasifikavimo uždavinį, tektų gilintis į klasifikuojamus objektus - pavyzdžiui, norint sukurti tam tikrų kiškių rūšių klasifikavimą, gilios žinios apie šias kiškių rūšių savybes būtų privalomos.

